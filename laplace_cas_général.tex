% Basic stuff
\documentclass{article}
\usepackage[utf8]{inputenc}
\usepackage[a4paper, total={6.5in, 9.5in}]{geometry}
\usepackage[bookmarks, hidelinks, unicode]{hyperref}
\usepackage[]{amsmath,amssymb}
\usepackage{stmaryrd}
\usepackage{tikz}
\usepackage{lmodern}
\usepackage{soul}
\usepackage{float}

% Packages configuration
\usetikzlibrary{shapes.arrows, angles, quotes}
\renewcommand{\arraystretch}{1.4}
\restylefloat{table}

% Shortcut commands
\newcommand{\im}{\text{Im}\,}
\newcommand{\re}{\text{Re}\,}
\newcommand{\img}{\text{Img}\,}
\newcommand{\R}{{\mathbb R}}
\renewcommand{\C}{{\mathbb C}}
\newcommand{\N}{{\mathbb N}}
\newcommand{\Z}{{\mathbb Z}}
\newcommand{\Q}{{\mathbb Q}}
\renewcommand{\U}{{\mathbb U}}
\newcommand{\cC}{{\mathcal C}}
\newcommand{\cD}{{\mathcal D}}
\newcommand{\cF}{{\mathcal F}}
\newcommand{\cotan}{\operatorname{cotan}}
\newcommand{\conj}[1]{\overline{#1}}
\newcommand{\Aff}{\text{Aff}}
\newcommand{\twoRows}[1]{\multirow{2}{*}{#1}}
\newcommand{\threeRows}[1]{\multirow{3}{*}{#1}}
\newcommand{\twoCols}[1]{\multicolumn{2}{c|}{#1}}
\newcommand{\threeCols}[1]{\multicolumn{3}{|c|}{#1}}
\newcommand{\twoColsNB}[1]{\multicolumn{2}{c}{#1}}
\newcommand{\goesto}[2]{\xrightarrow[#1\:\to\:#2]{}}
\newcommand{\liminfty}{\lim_{x\to+\infty}}
\newcommand{\limminfty}{\lim_{x\to-\infty}}
\newcommand{\limzero}{\lim_{x\to0}}
\newcommand{\const}{\text{cste}}
\newcommand{\et}{\:\text{et}\:}
\newcommand{\ou}{\:\text{ou}\:}
\newcommand{\placeholder}{\diamond}
\newcommand{\mediateur}{\:\text{med}\:}
\newcommand{\milieu}{\:\text{mil}\:}
\newcommand{\vect}[1]{\overrightarrow{#1}}
\newcommand{\point}[2]{(#1;\;#2)}
\newcommand{\spacepoint}[3]{\begin{pmatrix}#1\\#2\\#3\end{pmatrix}}
\newcommand{\sh}{\operatorname{sh}}
\newcommand{\ch}{\operatorname{ch}}
\renewcommand{\th}{\operatorname{th}}
\newcommand{\id}{\operatorname{id}}
\renewcommand{\cong}{\equiv}
\newcommand{\converges}[2]{\xrightarrow[{#1\to #2}]{}}
\newcommand{\convergedby}[2]{\xleftarrow[{#1\to #2}]{}}

% Document
\begin{document}

\section*{Dérivées et primitives de Laplace, cas général}


Soient $I \in \mathcal P(\R)$, $n \in \Z$ et $f \in \mathcal D^{n}(I, \R)$

\begin{align}
	\mathcal L [ f^{(n)} ] = p^{n}F(p)-\sum_{k=0}^{|n|-1} p^{n-k} f^{(k)}(0)
\end{align}

En notant, pour tout $n \in \Z\setminus\N$ et avec $A$, $B$ des ensembles:

\begin{enumerate}
	\item $\mathcal D^{n}(A, B)$ l'ensemble des fonctions de $A$ dans $B$ $n$ fois primitivables
	\item Pour tout $f \in \mathcal D^{n}(A, B)$, $f^{(n)}$ la fonction $n$-ième primitive de $f$ 
\end{enumerate}


\subsection*{Démonstration}
Montrons que le théorème précédent est vrai pour tout $n \in \N$ par récurrence.

\subsubsection*{Initialisation ($n = 0$)}
On a: 

\begin{align*}
	\mathcal L[f] =\mathcal L[f^{(0)}] &= p^0 F(p) - \sum_{k=0}^{|0|-1} p^{n-k} f^{(k)}(0) \\
						&= 1 \cdot F(p) - \sum_{\emptyset}^{} p^{n-k} f^{(k)}(0) \\
						&= F(p) \\
\end{align*}

\subsubsection*{Hérédité}
Soit $n \in \N$. Supposons (1).
D'après François Couprie, on sait que:

\begin{align}
	\mathcal L[f'] = \mathcal L[f^{(1)}] &= pF(p) - f(0)
\end{align}

On a donc

\begin{align*}
	\mathcal L[f^{(n+1)}] = \mathcal L[f^{(n)}'] &= pF^{(n)}(p) - f(0) \qquad \text{D'après (2)} \\
												 &= p\left( p^{n}F(p)-\sum_{k=0}^{|n|-1} p^{n-k} f^{(k)}(0)
												 \right) - f(0) \qquad \text{Par hypothèse de récurrence}\\
												 &= p^{n+1}F(p)-p \sum_{k=0}^{n-1} p^{n-k} f^{(k)}(0) - p^0f(0) \qquad \text{Car $n > 0$}\\
												 &= p^{n+1}F(p) - \sum_{k=0}^{n-1} p^{n-k+1} f^{(k)}(0) - \sum_{k=n}^{n} p^{n-k+1} f^{(k)}(0) \\
												 &= p^{n+1}F(p) -  \\
\end{align*}

\begin{flushright}
	{\footnotesize Ewen Le Bihan, MPSI Daudet, 2020}
\end{flushright}
\end{document}
